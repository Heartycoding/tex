\documentclass[]{ctexbook}
\usepackage{ctex}

\begin{document}
1、拉压刚度:$E \cdot A$ 扭转刚度: $ G \cdot I_p $ 弯曲刚度: $ E \cdot I $\\
$\lambda = \frac{\mu \cdot I}{ i }$ $E$ 为弹性模量,$ G $ 为剪切模量  $ A $ 为截面面积 , $ I_p $为极惯性矩, $ I $ 为惯性矩
,$ \mu $为长度因数 ,$ i $ 为惯性半径\\
2、$\sigma = E \cdot \epsilon$,$ \tau = G\cdot r$,$\sigma = \frac{E_y}{\rho} $; 拉压截面均匀分布,扭转沿截面径向向外线性增大,弯曲沿截面高度线性变化 \\
3、每个强度理论对应的相当应力\\
4、$ W_b = W_q + W_{BF}= 0 $


\end{document}